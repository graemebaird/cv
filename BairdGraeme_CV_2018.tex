
\documentclass{res}

\oddsidemargin -.5in
\evensidemargin -.5in
\textwidth=6.0in
\itemsep=0in
\parsep=0in
% if using pdflatex:
%\setlength{\pdfpagewidth}{\paperwidth}
%\setlength{\pdfpageheight}{\paperheight} 

\newenvironment{list1}{
	\begin{list}{\ding{113}}{%
			\setlength{\itemsep}{0in}
			\setlength{\parsep}{0in} \setlength{\parskip}{0in}
			\setlength{\topsep}{0in} \setlength{\partopsep}{0in} 
			\setlength{\leftmargin}{0.17in}}}{\end{list}}
\newenvironment{list2}{
	\begin{list}{$\bullet$}{%
			\setlength{\itemsep}{0in}
			\setlength{\parsep}{0in} \setlength{\parskip}{0in}
			\setlength{\topsep}{0in} \setlength{\partopsep}{0in} 
			\setlength{\leftmargin}{0.2in}}}{\end{list}}


\begin{document}
	
	
	\name{Graeme J. Baird \vspace*{.1in}}
	
	\begin{resume}
		\section{\sc Contact Information}
		\vspace{.05in}
		\begin{tabular}{@{}p{3in}p{4in}}
			1156 High St            &  {\it E-mail:}  gbaird@ucsc.edu \\            
			Department of Environmental Studies    \\         
			University of California, Santa Cruz &\\       
			Santa Cruz, CA 95064 USA  &  \\     
		\end{tabular}
		
		
		\section{\sc Research Interests}
		Agroecology, data science, Bayesian statistics, biogeochemistry, soil microbial ecology.		
		\section{\sc Education}
		{\bf University of California, Santa Cruz}, Santa Cruz, California USA\\
		\vspace*{-.1in}
		\begin{list1}
			\item[] Ph.D. Candidate, Environmental Studies 
			\begin{list2}
				\vspace*{.05in}
				\item Dissertation Topic:  ``Data-intensive approaches to 
				  ecological nutrient and disease management.'' 
				\item Advisor:  Carol Shennan
			\end{list2}
			\vspace*{.05in}
			\item[] M.A. with Honors, Environmental Studies, 2015
			\vspace*{.05in}
			\item[] B.A., Environmental Studies, 2010
		\end{list1}
		
		\section{\sc Research and Academic  Experience}
		{\bf University of California, Santa Cruz}, Santa Cruz, California USA
		
		\vspace{-.3cm}
		{\em Doctoral Candidate} \hfill {\bf September, 2013 - present}\\
		Includes current Ph.D.~research, coursework, and
		research projects.
	
		
		
				
		{\em Data analysis consultant} \hfill {\bf September, 2013 - present}\\
		Statistics and modeling consulting on a variety of projects in agronomy, entomology, ecology, agroecology, and sociology.
		
		{\em Graduate Student Researcher} \hfill {\bf September 2013 - present}\\
		Field research, data management, data analysis and modeling, and writing for USDA-funded projects related to disease control and organic systems management in CA Central Coast.
		
		%\vspace{-.1cm}
		{\em Lab manager and research technician} \hfill {\bf May 2010 - August 2013}\\
		Laboratory and field staff researcher. Duties included lab and field experiment setup, monitoring, and sampling, in addition to responsibility for in-house analysis of soil and plant samples via reagent-based and combustion-based analyses. Led field sampling, coordinated laboratory safety, trained and supervised employees in lab and field methods.     
		
	\vspace*{.05in}  
		
		{\bf Pontifical Catholic University of Valparaiso, Chile}, Valparaiso, Chile
		
		\vspace{-.3cm}		
		{\em Research Fellow} \hfill {\bf June 2017 - Dec 2017}\\
		Appointment via the competitive University of California fellowship "Research and Innovation Fellowship in Agriculture". Primary objectives were to serve as data analyst and statistician on a variety of projects, including: non-linear modeling of chill/heat requirements in cherry and prune crops in Chile's central valley; GLM and structural equation modeling of behavior and carbohydrate storage in almond spurs; dimension reduction techniques to investigate farmer practices and Phytophthora disease incidence in a country-wide survey of walnut production in Chile. 
				
		\vspace*{.05in}  
		
		{\bf Annie's Organic Foundation}, Oakland, California USA
		
		\vspace{-.3cm}		
		{\em Grant reader} \hfill {\bf 2016 - present}\\
		Grant reviewer for Annie's Organic "Sustainable Agriculture Fellowship" program. 
		

		\section{\sc Teaching Experience}
		{\bf University of California, Santa Cruz}, Santa Cruz, California USA
		
		\vspace{-.3cm}
		{\em Course leader - Introduction to Bayesian Modeling} \hfill {\bf Fall 2016}\\
		Organized and led a graduate-level student-run course on introductory Bayesian modeling and data analysis, using curriculum from the textbook "Statistical Rethinking". 
		
		
		\vspace{-.3cm}
		{\em Teaching Fellow} \hfill {\bf Fall 2016}\\
		Lecturer for \textit{Introduction to Agroecology}, ENVS 130A/L, an upper-division lecture class with 75 students and an associated mandatory field lab component. Class was partially based on existing curriculum with significant original modifications. I was responsible for all class content including lectures, organization and development of lesson plans and curriculum for field labs, and development of all assessment materials. 
		
		%\vspace{-.1cm}
		{\em Teaching Assistant} \hfill {\bf 2013 - 2017}\\
		Office hours, teaching classes, shared responsibility for grading and establishing class field experiment (130A/L). Held weekly field lab classes designed to introduce students to agroecology field methods (130L).
		\begin{list2}
			\item ENVS 133 Agroecology practicum, Spring 2014.
			\item ENVS 130A/L Introduction to Agroecology, Fall 2014.
			\item ENVS 130A/L Introduction to Agroecology, Fall 2015.		
			\item ENVS 100 Ecology and Society, Winter 2016.	
			\item ENVS 100 Ecology and Society, Winter 2017.
			\item ENVS 190 Capstone, Spring 2017.
		\end{list2}
		
		{\em Science Internship Project Mentor} \hfill {\bf Summer 2015; Summer 2016}\\
		Mentored high school interns in the competitive UCSC Science Intership Program. Interns worked on remote sensing and a multispectral imaging project for agricultural management. In these projects we developed a methodology, ran data collection flights, analyzed data, and produced findings. Students gained experience in a diverse set of skills including computer programming, field data collection, and statistical analysis. At the end of the fellowship worked to develop a final presentation and a formal submission to the Siemens STEM competition, one of which received an honorable mention award (2016).
		
				
		\section{\sc Publications}
		Baird, G., Luedeling, E., Alvarado, L., Fernandez, E., Cuneo, I., Bambach, N., Farias, D., and Saa, S. 2018. Nonlinear Ordinal Regression to Predict Bud Dormancy Requirements and Bud Burst in Deciduous Trees. Submitted to Frontiers in Plant Science. 
		
		Eldon, J., Baird, G., Sidibeh, S., Dobasin, D., Cheng, W., Shennan, C., Rapaport, P. 2018. On-farm trials identify adaptive management options for rainfed agriculture in West Africa. Submitted to Agricultural Systems.
		
		Fernandez, E., Baird, G., Farias, D., Oyanedel, E., Olaeta, J., Brown, P., Zwieniecki, M., Tixier, A., and Saa, S. 2018. Source/sink relationships in almond spurs define starch concentration and therefore their survival and bloom probabilities. Accepted to Scientia Horticulturae. 
		
		Shennan, C., Krupnik, T.J., Baird, G., Cohen, H., Forbush, K., Lovell, R.J. and Olimpi, E., 2018. Organic and Conventional Agriculture: A Useful Framing?. Annual Review of Environment and Resources.
		
		Shennan, C., Muramoto, J., Koike, S., Baird, G., Fennimore, S., Samtani, J., Bolda, M., Dara, S., Daugovish, O., Lazarovits, G. and Butler, D. 2018. Anaerobic soil disinfestation is an alternative to soil fumigation for control of some soil borne pathogens in strawberry production. Plant Pathology.
		
		Gianotti, AGS., Harrower, J., Baird, G., and Sepaniak, S. 2017. The quasi-legal challenge: Assessing and governing the environmental impacts of cannabis cultivation in the North Coastal Basin of California. Land Use Policy 61: 126-134.
		
		Muramoto, J., Shennan, C., Zavatta, M., Baird, G., Toyama, L.,  Mazzola, M. 2016. Effect of Anaerobic Soil Disinfestation and Mustard Seed Meal for Control of Charcoal Rot in California Strawberries. International Journal of Fruit Science, 16(1), 59-70.
		
		Shennan, C., J. Muramoto, G. Baird, M. Zavatta, L. Toyama, M. Mazzola, and S.T. Koike. 2015. Anaerobic soil disinfestation (ASD): a strategy for control of soil borne diseases in strawberry production. Acta Horticulturae (ISHS).
		
		Muramoto, J., Shennan, C., Zavatta, M., Baird, G., Toyama, L., Mazzola, M. 2015. Effect of Anaerobic Soil Disinfestation and Mustard Seed Meal for Control of Charcoal Rot in California Strawberries. International Journal of Fruit Science.
		
		Shennan, C., Muramoto, J., Baird, G., Zavatta, M., Toyama, L., Nieto, D., Bryer, J., Gershenson, A., Los Huertos, M., Kortman, S. and Klonsky, K., 2015, June. CAL-collaborative organic research and extension network: on-farm research to improve strawberry/vegetable rotation systems in coastal California. In International Symposium on Innovation in Integrated and Organic Horticulture (INNOHORT) 1137 (pp. 283-290).
		
		Muramoto, J., Shennan, C., Baird, G., Zavatta, M., Koike, S.T., Daugovish, O., Bolda, M.P., Dara, S.K., Klonsky, K., Mazzola, M. 2014. Optimizing Anaerobic Soil Disinfestation for California Strawberries. Acta Horticulturae 1044:215-220
		
		Zavatta, M., Shennan, C., Muramoto, J., Baird, G, Koike, S.T., Bolda, M.P., Klonsky, K. 2014. Integrated Rotation Systems for Soilborne Disease, Weed and Fertility Management in Strawberry/Vegetable Production. Acta Horticulturae 1044:269-274
		
		Shennan, C., Muramoto, J., Baird, G., Koike, S.T., Bolda, M.P., and Klonsky, K. 2012. Integrated rotation systems for soil borne disease, weed and fertility management in strawberry/vegetable production. Abstract for poster presentation. The 2nd International Organic Fruit Research Symposium, June 18-21, 2012. Leavenworth, WA.
		
		Shennan, C., Muramoto, J., Baird, G., Fennimore, S., Koike, S.T., Bolda, M.P., Daugovish, O., Dara, S., Mazzola, M., Lazarovits, G. 2012. Non-fumigant strategies for soilborne disease control in California strawberry production systems. Proceedings for the Annual International Conference on Methyl Bromide Alternatives and Emission Reductions, 16-1 – 16-4, Nov. 6-8, 2012. Orlando, FL.
		
		Shennan, C., Muramoto, J., Baird, G., Daugovish, O., Koike, S.T., Bolda, M.P. 2011.
		Anaerobic soil disinfestation: California. Proceedings of the Annual International Conference on Methyl Bromide   
		
		\section{\sc Papers in preparation}
		Baird G., Saa S., Guajardo J., Morales J., Alvarado L., Peach-Fine E., Canales X. 2018. Clusters of growers predict attitudes and behaviors towards on-farm management of Phythophthora rot and technology adoption in Chilean walnut production. In preparation for Agricultural Systems.

		\section{\sc Conference presentations}
		
		Baird, G, J. Muramoto, M. Zavatta, L. Toyama, and C.Shennan. Oct 2017. Cover Crops for Integrated Fertility Management in Organic Strawberry/Vegetable Production Systems. Oral Presentation at the 2017 Agronomy Society of America annual meeting, Tampa, FL. 
		
		Baird, G, J. Muramoto, M. Zavatta, and C.Shennan. Nov 2015. Factors Influencing Soil Inorganic Nitrogen Levels in an Organic Vegetable Cropping Rotation Study. Poster session presented at the Soil Science Society of America annual meeting, Minneapolis, MN.

		\section{\sc Honors and Awards} 
		UC Santa Cruz: Regents Fellowship, 2014, 2015; ENVS Departmental Research Award, 2014, 2015, 2016, 2017. Heller Agroecology Research Grant 2016. Heller Undergraduate Hire Agroecology Research Grant 2016; Outstanding TA Award 2017; Research and Innovation Fellowship in Agriculture 2017. 

		\section{\sc Service} 		
		UCSC Graduate Student Association representative for Environmental Studies. September 2016 - June 2017. 
		
		Graduate student representative on: Campus-wide Committee on Planning and Budget, September 2016 - June 2017;  Bookstore Planning Committee, March - June 2017; On-farm Research Committee for UCSC Center for Agroecology and Sustainable Food Systems, September 2016 - Present; Environmental Studies Graduate Steering Committee, September 2015 - June 2017; Environmental Studies faculty search committee for Stephen R. Gliessman Presidential Chair in Water Resources, January 2016.
		
	\end{resume}
\end{document}




